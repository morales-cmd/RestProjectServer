\documentclass[11pt]{article}   % list options between brackets
\usepackage[utf8]{inputenc}
\usepackage[T1]{fontenc}
\usepackage{lmodern}
\usepackage[polish]{babel}
\usepackage{amsmath}
\usepackage{blindtext}
%\usepackage{amsfonts}
%\usepackage{amssymb}
\usepackage{subcaption}
\usepackage{float}
\usepackage{wrapfig}
\usepackage{graphicx} % Allows including images
%\usepackage{booktabs} % Allows the use of \toprule, \midrule and \bottomrule in tables

% type user-defined commands here

\begin{document}
	
	\title{%
		Projekt końcowy\\
		\large Wprowadzenie do badań naukowych}
	\author{Paweł Marczak i Łukasz Kosmaty}         % type author(s) between braces
	\date{\today}    % type date between braces
	\maketitle
	
	
	\section*{Zadanie 1: \normalsize{Przeczytaj załączone krótkie streszczenie publikacji. Przeanalizuj je krytycznie i odpowiedz na
		pytania zamieszczone na końcu.}}     % section 1.1
	%\subsection{History}       % subsection 1.1.1
	\begin{itemize}
		\item Zakładając, że statystyki dotyczące omdleń na fotelach dentystycznych są takie jak podał
		dr McGimpsey, czy można zaakceptowac jego wyjaśnienie dotyczące śniadania lub
		wcześniejszego wyeksponowania na cierpienie?
	\end{itemize}

\par Należy odnieść się osobno do obu hipotez dr McGimpsey'a.
\par Hipoteza dotycząca niejedzenia śniadań wydaje się być uzasadniona, zważając na fakt, że dietetycy już stosunkowo dawno stwierdzili, że śniadanie jest bardzo ważnym posiłkiem w kontekście całego dnia i lekceważenie go może skutkować różnymi reperkusjami zdrowotnymi, m.in. omdleniami. Jak można przeczytać w różnych \'zródłach (https://www.medonet.pl/choroby-od-a-do-z/najczestsze-objawy-chorobowe,omdlenie,artykul,1578955.html), omdlenie to chwilowa utrata przytomności (zaburzenie pracy mózgu), która może być spowodowana zarówno strachem i bólem (wizyta u dentysty), jak i niedotlenieniem mózgu na skutek niedożywienia organizmu. \par Naturalnym wydaje się, że bycie niedożywionym oraz wystawionym na ból i strach zwiększa prawdopodobieństwo omdlenia, możnaby jednak przeprowadzić dokładniejsze badania, jak koegzystencja obu tych warunków wpływa na szanse omdlenia w stosunku np. do sytuacji, w której zachodzi tylko jeden z nich. Należy zaznaczyć jeszcze, że hipoteza dr McGimpsey'a ma sens, tylko wtedy, gdy mamy podstawy, by twierdzić, że mężczy/'zni faktycznie częsciej lekceważą śniadania.
\par Odnośnie zwiększonej ekspozycji na ból, ciężko jest autorom się odnieść do tego, jakie są różnice w odbiorze bólu u obu płci (z przyczyn oczywistych nie mają do czego przyrównać swoich wrażeń). Na podstawie dostępnych informacji, faktycznie można założyć, że kobiety silniej odczuwają ból i z tego powodu ich organizmy lepiej się do niego adaptują (https://www.npr.org/sections/health-shots/2019/08/26/741926952/women-may-be-more-adept-than-men-at-discerning-pain?t=1619201366942). Z tego powodu, ta hipoteza również wydaje się być prawdopodobna. Nawet sama częsta obserwacja bólu może mieć wpływ na prawdopodobieństwo omdlenia- obserwacja bólu może redukować strach przed nim, a jak wspomniano, strach może być jedną z przyczyn omdlenia.
\par Ciężko jest powiedzieć coś więcej odnośnie hipotez na podstawie samego streszczenia. Warto przeczytać całą publikację dr McGimpsey'a ze szczegółami, by ocenić trafność jego argumentacji.

\begin{itemize}
	\item Czy są jakieś prawdopodobne alternatywne hipotezy wyjaśniające różnicę w proporcjach
	omdleń na fotelu dentystycznym pomiędzy mężczyznami i kobietami?
\end{itemize}
\par Naturalnie, można wskazać inne możliwe przyczyny różnicy wska\'znika w obu grupach. Załóżmy, że mężczy\'zni gorzej dbają o higienę (jest to dość popularnym przeświadczeniem), co za tym idzie o higienę uzębienia. Prawdopodobnie oznaczałoby to, że częściej borykają się z poważniejszymi problemami jamy ustnej. Takie problemy wymagają zazwyczaj bardziej skomplikowanych i bolesnych zabiegów, a takie zabiegi częściej mogą powodować omdlenia. Wydaje się być to sensowną teorią, lecz wymagałaby sprawdzenia przyjętych założeń. Możnaby zacząć od sprawdzenia, czy faktycznie pacjenci częściej cierpią na poważne dolegliwości od pacjentek.
\par Inna ciekawa teoria, jest związana z podejściem do pacjentów samych stomatologów. W społeczeństwie często przypisuje się dentystom zapędy sadystyczne. Gdyby faktycznie założyć, że istotna część dentystów lubuje się w zadawaniu bólu, można spodziewać się, że praktykowaliby swoje hobby na pacjentach. Są też pewne powody by twierdzić, że byliby mniej powściągliwi w znęcaniu się nad pacjentami płci męskiej. W społeczeństwie panuje przekonanie, że znęcać się nad kobietami zwyczajnie nie wypada i ogólnie dręczenie kobiet jest bardzo \'zle odbierane (gorzej niż dręczenie mężczyzn). Oczywiście teoria ta wymaga wielu wątpliwych założeń i zdaniem autorów nie powinna być traktowana równie poważnie co pozostałe, lecz nie jest ona całkiem nieprawdopodobna. 
\par Kolejnym wyjaśnieniem mogą być zwyczajnie błędy badawcze, jak \'zle dobrane (nielosowe) dane, niewystarczająca ilość danych lub \'zle dobrany wska\'znik (w streszczeniu nie ma szczegółowych informacji dotyczących wsk\'znika). Być może wyjaśniane zjawisko nie ma miejsca w rzeczywistości, a poczyniono błędy w obserwacjach (nawet szanowani doktorowie lub ich podwładni czasami popełniają błędy). Ponownie, należy dokładniej zweryfikować publikację.
\section*{Zadanie 2: \normalsize{Choroba COVID-19, wywoływana przez koronawirusa, spowodowała duże zamieszanie na
	świecie. Jest oczywistym, że nauka uczyła się jak najlepiej jest reagować na nią w trakcie
	trwania pandemii, często metodą prób i błędów, o czym świadczy wysoka ilość zgonów w
	niektórych krajach. Proszę o zaproponowanie najskuteczniejszych, Państwa zdaniem,
	metodologii naukowych do wsparcia społeczeństw (i ich rządów!) w walce z pandemią}} 
\par Kluczowe w walce z pandemią są badania farmakologiczne (nad lekami i szczepionkami). Niestety autorom ciężko odnieść się do metodyki prowadzena takich badań, gdyż są one mocno oddalone od zakresu ich zainteresowań. Nie ulega wątpliwości jednak, że takie badania często wymagają wsparcia informatycznego w różnych zakresach,podobnie jak większość projektów badawczych w dzisiejszych czasach.
\par Są jednak inne dziedziny walki z pandemią, jak na przykład modelowanie i symulowanie jej rozwoju. Ta dziedzina przykuwa szczególną uwagę rządów państw opracowujących strategię przeciwdziałania rozprzestrzenianiu się wirusa. W tym obszarze, umiejętności autorów mogą się okazać nieco bardziej przydać.
\par  Standardową praktyką jest kolekcjonowanie danych dotyczących wirusa (liczba wykrytych przypadków, ozdrowieńców, zgonów w określonym czasie i miejscu). Wykorzystując metody statystyczne, na podstawie zebranych danych można utworzyć model zachowywania krzywej zakażeń i na jej podstawie prognozować jaki będzie rozwój wypadków. Model taki może być bardzo przydatny przy planowaniu momentu wprowadzenia wybranych obostrzeń lub weryfikacji, czy już wprowadzone obostrzenia przynoszą oczekiwane rezultaty.
\par Potężnym narzędziem w walce z pandemią wydają się być symulacje komputerowe. Ciekawy film będący wprowadzeniem w symulowanie pandemii można zobaczyć pod adresem: \newline https://www.youtube.com/watch?v=gxAaO2rsdIs\&t=1s \newline
Autor filmu pokazuje na czym polega taka symulacja i jak obszerne wnioski mogą z niej wynikać. Prawdą jest, że przyjmuje on na potrzeby testowania wymyślone wartości istotnych parametrów, jednak nic nie stoi na przeszkodzie by dobierać te wartości na podstawie rzeczywistych danych (na przykład wykorzystać wspomnianą analizę statystyczną). Można też zmodyfikować projekt podsunięty przez twórcę, by bardziej odpowiadał on specyfice koronawirusa. Dobrze zaprojektowana symulacja pozwalałaby w jakimś stopniu przewidzieć, jak rozmaite manipulacje ze strony rządów wpływałyby na sytuację pandemiczną i obrać optymalną ścieżkę działania. Oprócz tego, taką sztuczną pandemię można wykorzystać przy edukowaniu społeczeństwa (sam cytowany filmik autorzy uważają za dość pouczający, pomimo tego, że nie został poparty twardymi faktami dotyczącymi koronawirusa).
\section*{Zadanie 3: \normalsize {Szczepionka AstraZeneca przeciw COVID-19 spotkała się z nieufnością, krytyką, a nawet z
	zawieszeniem jej stosowania w niektórych krajach, z powodu przypadków zakrzepicy u
	pacjentów po otrzymaniu szczepionki. Znajdź wiarygodne statystyki i inne informacje naukowe
	na temat ryzyka związanego z tą szczepionką i poddaj je krytycznej analizie.}} 
	



\end{document}
